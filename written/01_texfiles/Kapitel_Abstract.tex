\addchap*{Abstract}
Die vorliegende Arbeit befasst sich mit dem Entwurf eines Spurhalteassistenten f\"ur ein automatisiertes Modellfahrzeug. Die Spurf\"uhrung soll, bei einer konstant eingestellten Geschwindigkeit, durch das Assistenzsystem \"ubernommen werden. F\"ur die notwendige Spurerkennung wird eine Kamera verwendet.\\
\\
In dieser Arbeit wird der gesamte Entwicklungsprozess des Fahrerassistenzsystems abgebildet. Der Schwerpunkt der Arbeit liegt in einer industrienahen Entwicklung. Daf\"ur werden Entwicklungsmethoden und -tools verwendet, die auch in der Industrie zum Einsatz kommen. Die Spurerkennung arbeitet mit einem selbst entwickelten Linienerkennungs- und Spurklassifizierungsalgorithmus. Um die Querregelung durchzuf\"uhren, wird ein Regelungskonzept verwendet, das aus kaskadierten PD-Reglern besteht. Die Regelung wird zun\"achst simluativ an einem Fahrzeugmodell getestet.\\
\\
Abschlie�end wird die Regelung auf dem realen Fahrzeug implementiert und getestet. Dabei wird eine Integration des Assistenzsystems gew\"ahlt, die dem Fahrer Lenkeingriffe erm\"oglichen. Das Ergebnis der Arbeit ist, dass das entwickelte Fahrerassistenzsystem in der Lage ist, die Spurf\"uhrung selbstst\"andig durchzuf\"uhren.

