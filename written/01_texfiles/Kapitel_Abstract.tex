\addchap*{Abstract}
Data-driven systems in areas such as autonomous driving or predictive maintenance are increasingly based on the processing of time series data that are collected by the increasing amount of sensors. For finding patterns in these data, pattern recognition algorithms are applied. However, most of these algorithms do not use the original time series as input data, but a more suitable representation of them. Such a representation can be obtained by discretizing the time series in a preprocessing step based on a time series discretization approach. For an efficient and effective pattern finding, such a time series discretization approach shall provide a compressed representation of the original time series, while preserving its features. \newline
In this thesis, five time series discretization approaches are evaluated. First, for each time series discretization approach, its goodness with respect to a feature-preserving discretized representation of the original time series is measured. Second, the applicability of these approaches with respect to motif discovery, which is the task of finding recurrently occurring patterns in a time series, is evaluated. For this evaluation, each time series discretization approach is used in a preprocessing step to obtain discretized time series representations that represent the input data of three motif discovery algorithms. Moreover, for two of these three algorithms, the results based on applying a time series discretization approach in a preprocessing step are compared to the results when inputting the original time series without discretization. \newline
The experimental results indicate that two of the five evaluated time series discretization approaches are superior with respect to a feature-preserving discretized representation. Moreover, the results for the employed motif discovery algorithms indicate that four time series discretization approaches perform similar, while the remaining approach performs worse. Overall, the results imply that those four approaches are suitable as a preprocessing step for the employed motif discovery algorithms.

