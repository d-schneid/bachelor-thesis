\section{Motivation}
The influence of time series data on data-based decisions has increased enormously in the past. Due to the advancing digitization, this trend will continue in the future \cite{Forbes}. Data-driven systems in areas such as autonomous driving or predictive maintenance are increasingly based on the processing of time series data. For example, a fully autonomous vehicle is estimated to generate 19 terabytes of data per hour that are generated by its sensors \cite{Sensor_Data}. As a result of this increase in time series data processing, the demand for optimized databases for collecting and analyzing this type of data has also risen sharply in recent years \cite{TS_Databases}. \newline
The advantage of analyzing time series data is the ability to identify changes and patterns that develop over time \cite{Survey_Esling}. To exploit this advantage, it is necessary to develop effective and efficient algorithms for processing and analyzing time series data. A prerequisite for such algorithms is meaningful input data, which can be obtained by a suitable representation of the time series \cite{Survey_Esling}. Based on the state of the art of science, the discretization of time series represents a prominent approach in this respect \cite{Survey_Esling}. This motivates the evaluation of different discretization approaches for the representation of time series in the context of pattern recognition.
\section{Problem Statement}
As most algorithms for time series pattern recognition do not use the original time series as input data, a preprocessing step is applied in which the time series is converted into a more suitable representation \cite{Survey_Esling}. The first objective of such a preprocessing step is to compress the time series data in order to reduce the time and memory requirements of the pattern recognition algorithms \cite{Survey_Esling}. The other objective is to transform the time series into a representation that reflects its characteristic features \cite{Survey_Esling}. \newline
Thus, the difficulty when using discretization as such a preprocessing step is to preserve and extract the characteristic features of the time series as accurately as possible, while removing less informative data such as noise. By preserving the characteristic features, the results of pattern recognition algorithms based on the corresponding representation can be reliably transferable to the original time series. Hence, discretization approaches for time series have to be found that fulfill two properties. First, for effectively applying pattern recognition algorithms, they shall provide a feature-preserving representation of the original time series. Second, for efficiently applying pattern recognition algorithms, they shall provide a compressed representation of the original time series.
\section{Aim of the Thesis}
In this thesis, existing discretization approaches for the representation of time series shall be evaluated. For this purpose, suitable metrics shall be used to quantify and evaluate the goodness of these approaches with respect to a feature-preserving representation of the original time series. \newline
Furthermore, the applicability of the examined discretization approaches as a preprocessing step for time series pattern recognition algorithms shall be evaluated. The focus of time series pattern recognition shall be the identification of recurrently occurring patterns in a time series, which is called Motif Discovery \cite{Survey_Esling}.
\section{Structure of the Thesis}
Chapter \ref{first_chapter} introduces fundamental concepts that are used in this thesis. In Chapter \ref{chap:ts_discretization}, the evaluated discretization approaches for time series are presented. Chapter \ref{subsection_motif_discovery} presents the pattern recognition algorithms with respect to motif discovery. These algorithms are used to evaluate the applicability of the examined time series discretization approaches as a preprocessing step. Chapter \ref{evaluation_chapter} contains the results of the experimental evaluation of the examined time series discretization approaches. In Chapter \ref{last_chapter}, a summary of the findings of this thesis along with an outlook is presented.


