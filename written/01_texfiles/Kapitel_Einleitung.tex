\chapter{Einleitung und Motivation}
\section{Einleitung}
\begin{quote} Wer nur an die Technik denkt, hat noch nicht erkannt, wie das autonome Fahren unsere Gesellschaft ver\"andern wird. \\(Dr. Dieter Zetsche, Vorstandsvorsitzender der DaimlerAG, 2005)
 \end{quote}
F\"ur die heutige Gesellschaft nimmt die Bedeutung der Mobilit\"at immer weiter zu. Von Arbeitnehmern wird eine immer weiter steigende Flexibilit\"at, bez\"uglich ihres Arbeitsplatzes, gefordert. Das f\"uhrt zu einer kontinuierlichen Erh\"ohung des Verkehrsaufkommens und der durchschnittlichen Zeit, die ein Arbeitnehmer im Auto verbringt. Im Jahr 2008 nutzten mehr als zwei Drittel der Arbeitnehmer das Auto, um zur Arbeit zu gelangen. Die durchschnittliche Fahrzeit, f\"ur den Weg zur Arbeit und zur\"uck, betrug dabei 90 Minuten \cite{Radstand2008}. Hierdurch nimmt die Belastung der Arbeitnehmer immer weiter zu. Dies kann sowohl physische Belastung, wie R\"uckenschmerzen aufgrund der Sitzhaltung, als auch psychische Belastung sein. Beispielsweise f\"uhrt der, durch einen Stau oder hohes Verkehrsaufkommen entstandene, Zeitdruck zu einer hohen psychischen Belastung. Besonders in Zeiten, in denen auf Grund der wachsenden Wirtschaft und Globalisierung der G\"uterverkehr zunimmt\cite{Huetter2013}, gewinnt das Thema an Bedeutung. In Deutschland wird rund 75 Prozent des gesamten G\"uterverkehrs auf der Stra{\ss}e transportiert. F\"ur den Zeitraum der n\"achsten 15 Jahre wird eine Zunahme von 40 Prozent prognostiziert\cite{NTV2016}. Eine weitere Branche die die Entwicklung automatisierter Fahrfunktionen beschleunigt ist die Fernbus-Branche. Der Anteil an Personenverkehr mit Fernbussen hat in den letzten Jahren stark zugenommen und auch hier ist die Tendenz steigend\cite{ODB2014}. Durch das steigende Verkehrsaufkommen, sowie Verkehrsunf\"allen und Staus, nimmt zudem die Belastung f\"ur die Umwelt zu. Durch die Entwicklung von Fahrerassistenzsystemen und automatisierter Fahrfunktionen soll sowohl der Fahrkomfort erh\"oht, als auch die Anzahl von Verkehrsunf\"allen und Staus reduziert werden. Das Zitat von Dr. Zetsche bezieht sich darauf, dass diese Zeit, die heutzutage in das F\"uhren des Autos investiert werden muss, in Zukunft anders genutzt werden kann. Durch die Entwicklung vollst\"andig autonomer Fahrzeuge kann es zu einer Ver\"anderung, sowohl in der Arbeitswelt, als auch im gesellschaftlichen Denken kommen. Da die Fahrzeit als vollwertige Arbeitszeit genutzt werden kann, verf\"ugen die Arbeitnehmer \"uber mehr Freizeit. Zudem werden neue Berufsm\"oglichkeiten geschaffen. Beispielsweise im G\"uterverkehr sind die Fahrer in der Lage andere Aufgaben zu erledigen. Dies k\"onnen beispielsweise Management- oder Servicet\"atigkeiten f\"ur ihre Logistikfirma oder Spedition sein. Auch die Wahl des Wohnortes ist dann nicht mehr so stark an den Standort der Arbeit gebunden. So k\"onnen Ballungsr\"aume reduziert werden und l\"andliche Gebiete k\"onnten wieder an Attraktivit\"at gewinnen.\cite{Rahmann2011}
\section{Motivation}
Ein aktuelles Projekt, am Forschungszentrum Informatik in Karlsruhe, ist der Aufbau des ESM\footnote{eingebettete Systeme und Mikrosysteme} Demonstrators. Mit ihm sollen aktuelle Forschungsprojekte, sowie zuk\"unftig auch Forschungsergebnisse, im Automotive-Bereich pr\"asentiert werden k\"onnen. Ziel ist die Entwicklung automatisierter Fahrfunktionen. Dabei wird besonderer Fokus auf den Entwicklungsprozess gelegt. Es werden verschiedene Herangehensweisen und Entwicklungsmethoden evaluiert. Um den Entwicklungsprozess industrienah durchzuf\"uhren, werden Tools verwendet, die auch in der Industrie zum Standard geh\"oren. Neben der Funktionsentwicklung ist auch die Absicherung der entwickelten Funktionen ein zentraler Aspekt des Demonstrators. Ziel dieser Masterarbeit ist die Entwicklung und der Test einer ersten automatisierten Fahrfunktion am Demonstrator. Als Fahrfunktion soll ein automatisierter Spurhalteassistent entwickelt werden. Wie die vermehrten Unf\"alle von \textit{Model S}, sowie die z\"ogerliche Einf\"uhrung automatisierter Fahrfunktionen durch andere Hersteller, zeigen, ist die Entwicklung automatisierter Fahrfunktionen noch nicht abgeschlossen. Aber auch andere Automobilhersteller, wie BMW mit der aktuellen 7er-Reihe, Volvo mit dem XC90, Audi mit dem A4 oder Mercedes mit der S-Klasse, werben bereits mit teilautonomen Fahrfunktionen. Die Entwicklung und vor allem die Absicherung automatisierter Fahrfunktionen wird in Zukunft immer mehr an Bedeutung gewinnen. Dieser Trend ist auch auf der internationalen Automobil Ausstellung IAA zu erkennen. Die Entwicklung automatisierter Fahrfunktionen r\"uckt, neben der Elektromobilit\"at und der Vernetzung, sowohl im Bereich der Personalkraftwagen, als auch bei den Nutzfahrzeugen, immer weiter in den Mittelpunkt. Aus diesem Grund wird auch am Forschungszentrum Informatik an der Entwicklung automatisierter Fahrfunktionen geforscht.
