\subsection{k-Means Clustering}
Given a set of observations $X = \{x_1, ..., x_N\} \subset \mathbb{R}^d \ (d, N \geq 1)$, the k-means clustering algorithm partitions $X$ into a given number of $1 \leq k \leq N$ disjoint sets $C_1, ..., C_k$, which are called clusters \cite{K_Means_Explanation}. These clusters shall minimize the \ac{SSE} \cite{K_Means_Explanation}:
\begin{equation}
\min_{C_1, ..., C_k} \ \sum_{i}^{k}\sum_{x \in C_i}D_2(x,\mu_i)^2,
\label{eq:objective_k_means}
\end{equation}
where $D_2$ is the Euclidean distance and $\mu_i \in \mathbb{R}^d \ (1 \leq i \leq k)$ is the mean of all points $x \in C_i \subseteq X$, which is also called the cluster center of $C_i$. For finding such minimizing clusters, the following procedure is applied \cite{K_Means_Explanation}:
\begin{enumerate}[label=\arabic*., ref=\arabic*, itemsep=0pt,parsep=5pt]
    \item randomly select $k$ points from $X$ as initial cluster centers
    \item repeat until $C_1, ..., C_k$ stop changing:
    \begin{enumerate}[label*=\arabic*, ref=\theenumi.\arabic*]
        \item assign each $x \in X$ to the closest cluster $C_i$ with respect to the Euclidean distance between $x$ and $\mu_i$
        \item for each cluster $C_i$ compute its new cluster center as the mean of all assigned points $x \in C_i$
    \end{enumerate}
\end{enumerate}
