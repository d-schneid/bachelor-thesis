\chapter{Motif Discovery} \label{subsection_motif_discovery}
Motif discovery is the task of finding all recurrently occurring subsequences in a given time series \cite{Survey_Esling}. More formally, let $X = x_1, ..., x_N$ be a time series of length $N \ge 1$. Then, the task of motif discovery is to find all subsequences $S = s_1, ..., s_M$ of length $1 \leq M < N$ that recurrently occur in $X$ (see Figure \ref{fig:intro_motifs}) \cite{Survey_Esling}. \newline
The determination of recurrently occurring subsequences in $X$ involves a similarity measurement between subsequences based on a given similarity measure $D$ (e.g. the Euclidean distance) and a given similarity distance $r \geq 0$ \cite{Motif_Definitions}. Two subsequences $S_1$ and $S_2$ of $X$ are then called matching subsequences if and only if \mbox{$D(S_1,S_2) \leq r$} \cite{Motif_Definitions}. Moreover, trivially matching subsequences of $X$ are defined as matching subsequences whose starting points in time are not further apart than a given $k \geq 0$ points in time \cite{Motif_Definitions}. Based on matching subsequences, recurrently occurring subsequences in $X$ can be found as explained for the following motif discovery algorithms. Further note that the recurrent occurrence of a subsequence in a given time series is relative to the given similarity measure and the given similarity distance.
\begin{figure}[htb]
\centering
\includegraphics[width=0.8\textwidth]{pattern_recognition/motif_discovery/motif_discovery.pdf}
\caption[Motif Discovery - Recurrently Occurring Subsequence]{The three plotted subsequences of the original standardized time series show a similar pattern. Hence, this pattern represents a recurrently occurring subsequence in the original standardized time series \cite{Motif_Definitions}.}
\label{fig:intro_motifs}
\end{figure}
\newpage
\section{Random Projection} \label{description_random_projection}
Let $X = x_1, ..., x_N$ be a standardized time series of length $N \geq 1$. Then, the first step of the Random Projection motif discovery algorithm is to extract subsequences of $X$ with an overlapping sliding window of a given length $1 \leq l \leq N$ (see Subfigure \ref{fig:extract_subsequences}) \cite{Random_Projection}. These extracted subsequences are then individually discretized based on one of the time series discretization algorithms described in Chapter \ref{chap:ts_discretization}. The selected time series discretization algorithm is employed with the same parameters for each extracted subsequence. This also includes that the adaptive breakpoints used for discretization in the \ac{aSAX} and Persist are not computed individually for each extracted subsequence, but for the whole standardized time series $X$. Thus, the discretized subsequences are comparable with each other. \newline
Based on these discretized subsequences, a matrix $\hat{M}_{p \times q} \ (p,q \geq 1)$ is then created (see Subfigure \ref{fig:discretized_matrix}) \cite{Random_Projection}. In each row, this matrix contains one of the discretized subsequences, while in each column it contains one of the alphabet symbols of the respective discretized subsequence. As each subsequence of $X$ is extracted and discretized based on the same parameters, they all consist of the same number of alphabet symbols. Moreover, the row-wise order of the discretized subsequences within $\hat{M}_{p \times q}$ corresponds to their order of extraction. Such that the discretized subsequence that was extracted first is contained in the first row and the discretized subsequence that was extracted last is contained in the last row. \newline
Based on the constructed matrix $\hat{M}_{p \times q}$, the random projection procedure is employed \cite{Random_Projection}. First, a given number of $0 \leq s < q$ columns are randomly selected. These randomly selected columns act as a mask as they are hidden and modify $\hat{M}_{p \times q}$ to $\hat{M}_{p \times q'}$ with $q' := q-s$ (see Subfigure \ref{fig:random_projection_matrix}). Second, a collision matrix $C_{p \times p}$ is created that contains a row and a column for each of the $p$ discretized subsequences (see Subfigure \ref{fig:collision_matrix}). In cell $(i,j) \ (1 \leq i,j \leq p)$, this collision matrix contains the number of collisions of the remaining alphabet symbols of the discretized subsequences $i$ and $j$ in $\hat{M}_{p \times q'}$, when performing the random selection of $s$ columns $iters \geq 1$ times on $\hat{M}_{p \times q}$. The discretized subsequences $i$ and $j$ collide if their alphabet symbols are equal for each column of $\hat{M}_{p \times q'}$. Note that $C_{p \times p}$ is a symmetric matrix. Further note that for the \ac{1d-SAX} and {eSAX}, $s$ is the number of two respectively three adjacent columns as they use two respectively three alphabet symbols for discretizing a subsequence extracted by the \ac{PAA}. \newline
Based on the collision matrix $C_{p \times p}$, matching subsequences are likely to have a relatively high number of collisions and a relatively high number of collisions indicates matching subsequences \cite{Random_Projection}. Therefore, the next step is to use the collision matrix $C_{p \times p}$ as a filter that indicates which extracted subsequences of $X$ shall be examined for finding matching subsequences and which not \cite{Random_Projection}. First, the extracted subsequences that correspond to the largest number of collisions in $C_{p \times p}$ are retrieved. Let $S_{1}^*$ and $S_{2}^*$ be those subsequences. Then, the Euclidean distance $D_2(S_{1}^*,S_{2}^*)$ is computed and compared to a given similarity distance $r \geq 0$. If $D_2(S_{1}^*,S_{2}^*) \leq r$ holds, $S_{1}^*$ and $S_{2}^*$ are matching subsequences and form a so called tentative motif. This tentative motif is expanded to a motif by adding each other extracted subsequence of $X$ that is a matching subsequence to $S_{1}^*$ or $S_{2}^*$. \newline
To find those other extracted subsequences while avoiding costly memory accesses, the collision matrix $C_{p \times p}$ is used as a filter again \cite{Random_Projection}. Only those extracted subsequences of $X$ that correspond to discretized subsequences whose number of collisions with the discretized subsequences corresponding to $S_{1}^*$ or $S_{2}^*$ is above a given threshold $min \textunderscore collisions \geq 1$ are retrieved. Let $S_3$ be such an extracted subsequence, then it is included into the tentative motif formed by $S_{1}^*$ and $S_{2}^*$ if and only if it is a matching subsequence to $S_{1}^*$ or $S_{2}^*$ (i.e. $D_2(S_{1}^*,S_{3}) \leq r$ or $D_2(S_{2}^*,S_{3}) \leq r$). The final motif is then the one that contains $S_{1}^*$ and $S_{2}^*$ along with all of their matching subsequences. This motif building procedure is done iteratively, while all extracted subsequences that are already included in a motif cannot be selected for future motifs \cite{Random_Projection}. In each iteration, the two extracted subsequences corresponding to the highest number of collisions in the current iteration are retrieved until all remaining numbers of collisions are below $min \textunderscore collisions$ or no more extracted subsequences are available to be included in a motif. \newline
Note that for the \ac{SAX}, even less extracted subsequences may need to be retrieved by applying the procedure based on the $MINDIST$ described in Subsection \ref{dist_measure_mindist} for discretized subsequences as an additional filter \cite{Random_Projection}.
\begin{figure}
\centering
\begin{subfigure}[b]{0.45\textwidth}
\includegraphics[width=\textwidth]{pattern_recognition/motif_discovery/random_projection/extract_subsequences.pdf}
\caption{Three exemplary subsequences of the standardized time series $X$ that are extracted by an overlapping sliding window.}
\label{fig:extract_subsequences}
\end{subfigure}
\hfill%
\begin{subfigure}[b]{0.45\textwidth}
    \centering
    \begin{tabular}{c|c|c|c|c|}
    \cline{2-5}
    0 & \textcolor{red}{b} & \textcolor{red}{a} & \textcolor{red}{b} & \textcolor{red}{c} \\
    ... & ... & ... & ... & ... \\
    40 & \textcolor{orange}{a} & \textcolor{orange}{a} & \textcolor{orange}{b} & \textcolor{orange}{b} \\
    ... & ... & ... & ... & ... \\
    85 & \textcolor{purple}{b} & \textcolor{purple}{b} & \textcolor{purple}{b} & \textcolor{purple}{a} \\
    \cline{2-5}
  \end{tabular}
    \caption{The matrix $\hat{M}_{p \times q}$ with $p = 86$ extracted subsequences that were discretized into $q = 4$ alphabet symbols.}
    \label{fig:discretized_matrix}
  \end{subfigure}
\\[10pt]
\begin{subfigure}[b]{0.45\textwidth}
    \centering
    \begin{tabular}{c|c|>{\columncolor{gray!25}}c|c|>{\columncolor{gray!25}}c|}
    \hhline{~|----|}
    0 & \textcolor{red}{b} &  & \textcolor{red}{b} & \\
    ... & ... & \phantom{...} & ... & \phantom{...} \\
    40 & \textcolor{orange}{a} & & \textcolor{orange}{b} & \\
    ... & ... &  & ... &  \\
    85 & \textcolor{purple}{b} & \phantom{...} & \textcolor{purple}{b} & \phantom{...} \\
    \hhline{~|----|}
  \end{tabular}
    \caption{For instance, the second and fourth column are randomly selected to be hidden in the first iteration. Then, the first and last discretized subsequence collide.}
    \label{fig:random_projection_matrix}
  \end{subfigure}
 \hfill%
\begin{subfigure}[b]{0.45\textwidth}
    \centering
    \begin{tabular}{c|c|c|c|c|c|}
	\multicolumn{1}{@{}c}{} & \multicolumn{1}{c}{0} & \multicolumn{1}{c}{...} & \multicolumn{1}{c}{40} & \multicolumn{1}{c}{...} & \multicolumn{1}{c}{85} \\    
    \hhline{~|-----|}
    0 & \cellcolor{gray!25}\phantom{...} & \cellcolor{gray!25} & \cellcolor{gray!25} & \cellcolor{gray!25} & \cellcolor{gray!25} \\
    \hhline{~|-----|}
    ... & ... & \cellcolor{gray!25}\phantom{...} & \cellcolor{gray!25} & \cellcolor{gray!25} & \cellcolor{gray!25} \\
    \hhline{~|-----|}
    40 & 0 & ... & \cellcolor{gray!25}\phantom{...} & \cellcolor{gray!25} & \cellcolor{gray!25} \\
    \hhline{~|-----|}
    ... & ... & ... & ... & \cellcolor{gray!25}\phantom{...} & \cellcolor{gray!25} \\
    \hhline{~|-----|}
    85 & 1 & ... & 0 & ... & \cellcolor{gray!25}\phantom{...} \\
    \hhline{~|-----|}
  \end{tabular}
    \caption{The symmetric collision matrix $C_{p \times p}$ with $p = 86$ after the first of $iters$ iterations. It contains the collision of the first and last discretized subsequence.}
    \label{fig:collision_matrix}
  \end{subfigure}
\caption[Random Projection - Steps and Data Structures]{Steps and data structures that are employed by the Random Projection motif discovery algorithm \cite{Random_Projection}.}
\label{fig:blablaal}
\end{figure}








\newpage
\section{Matrix Profile} \label{description_matrix_profile}
Let $X = x_1, ..., x_N$ be a standardized time series of length $N \geq 1$. Then, the first step of the Matrix Profile motif discovery algorithm is to discretize $X$ based on one of the time series discretization algorithms described in Chapter \ref{chap:ts_discretization}. Let $\hat{X}$ be the resulting discretized time series corresponding to $X$. The second step is to encode the alphabet symbols that represent $\hat{X}$ based on the alphabet size $a \geq 2$ used for discretization. For this, the encoding $e(\alpha_j) = j-1 \ (1 \leq j \leq a)$ is applied on the alphabet symbols $\alpha_j$ that represent $\hat{X}$. Note that for the \ac{1d-SAX}, the discretized means and discretized slopes need to be encoded based on the size of the respective alphabet. Let $\hat{X}^e$ be the encoded discretized time series that results from applying the encoding on each alphabet symbol that represents $\hat{X}$. \newline
In the next step, the Matrix Profile procedure is applied on $\hat{X}^e$ \cite{Matrix_Profile}. With this procedure, the pairwise distances between any two subsequences of $\hat{X}^e$ based on a given length $l \geq 1$ of the subsequences and a given Minkowski distance $D$ are computed. Compared to other procedures that compute such pairwise distances (e.g. brute force), the Matrix Profile procedure provides a more efficient computation \cite{Matrix_Profile}. However, it only works on numerical data, which is the reason $\hat{X}$ needs to be encoded. \newline
Based on the computed pairwise distances between any two subsequences of $\hat{X}^e$, the pairwise mutually nearest neighbors are computed. Two subsequences $S_1$ and $S_2$ of $\hat{X}^e$ are pairwise mutually nearest neighbors if both have their smallest distance to the respective other according to the computed pairwise distances. All pairwise mutually nearest neighbors are then sorted in an ascending order with respect to the pairwise distance. \newline
Based on this sorting, the idea of the Random Projection motif discovery algorithm described above is applied \cite{Random_Projection}. Matching subsequences of $X$ are likely to have a relatively small pairwise distance. Therefore, the next step is to use this sorting as a filter that indicates which subsequences of $\hat{X}^e$ shall be examined for finding matching subsequences of $X$ and which not. First, the two subsequences of $\hat{X}^e$ that represent the pairwise mutually nearest neighbors with the smallest pairwise distance are retrieved. Let $S_{1}^*$ and $S_{2}^*$ be those subsequences. If $D(S_{1}^*,S_{2}^*) \leq r$ holds for a given similarity distance $r \geq 0$, then $S_{1}^*$ and $S_{2}^*$ are declared a tentative motif. This tentative motif is expanded to a motif by adding each other subsequence $S_3$ with length $l$ of $\hat{X}^e$ that fulfills $D(S_{1}^*,S_{3}) \leq r$ or $D(S_{2}^*,S_{3}) \leq r$. Such a subsequence $S_3$ is found by computing the pairwise distance between any other subsequence with length $l$ of $\hat{X}^e$ and $S_{1}^*$ respectively $S_{2}^*$ based on the Matrix Profile procedure \cite{Matrix_Profile}. \newline
The final motif is then the one that contains $S_{1}^*$ and $S_{2}^*$ along with all other subsequences $S_3$ with length $l$ of $\hat{X}^e$ that fulfill $D(S_{1}^*,S_{3}) \leq r$ or $D(S_{2}^*,S_{3}) \leq r$ (see Figure \ref{fig:two_circles}) \cite{Random_Projection}. Based on such a final motif, the corresponding subsequences of $X$ can be retrieved. This motif building procedure is done iteratively, while all subsequences of $\hat{X}^e$ that are already included in a motif cannot be selected for future motifs. In each iteration, the two subsequences of $\hat{X}^e$ that represent the pairwise mutually nearest neighbors with the smallest pairwise distance in the current iteration are retrieved until no more subsequences of $\hat{X}^e$ are available to be included in a motif. \newline
Note that for the \ac{1d-SAX} and \ac{eSAX}, a subsequence of $X$ extracted by the \ac{PAA} transforms into two respectively three encoded alphabet symbols in $\hat{X}^e$. Therefore, assuming that for $\hat{X}^e$ the points in time start at zero, subsequences of $\hat{X}^e$ that are used for pairwise distance computations should start at a point in time that is divisible by two respectively three. Moreover, $l$ should also be divisable by two respectively three. These two actions ensure comparability between the subsequences of $\hat{X}^e$ when computing pairwise distances. Otherwise, they would not correspond to the subsequences of $X$ extracted by the \ac{PAA} for the \ac{1d-SAX} and \ac{eSAX}.
\begin{figure}[htb]
\centering
\includegraphics[width=0.8\textwidth]{pattern_recognition/motif_discovery/matrix_profile/two_circles.pdf}
\caption[Matrix Profile - Motif]{A visual intuition for a motif that is discovered by the Matrix Profile motif discovery algorithm \cite{Random_Projection}. In this plot, $D(S_{1}^*,S_{2}^*) = r$ is indicated as an example. Moreover, the red dots indicate subsequences $S_3$ of $\hat{X}^e$ that fulfill $D(S_{1}^*,S_{3}) \leq r$ or $D(S_{2}^*,S_{3}) \leq r$ and are therefore included in the motif.}
\label{fig:two_circles}
\end{figure}








\section{Brute Force} \label{description_brute_force}
Let $X = x_1, ..., x_N$ be a standardized time series of length $N \geq 1$. The first two steps of the Brute Force motif discovery algorithm are the same as for the Matrix Profile motif discovery algorithm described above. Therefore, let $\hat{X}^e$ be the encoded discretized time series that results from applying the encoding on each alphabet symbol that represents the discretized time series $\hat{X}$. \newline
The next step is to extract subsequences of $\hat{X}^e$ with an overlapping sliding window of a given length $l \geq 1$ \cite{Motif_Definitions}. Analogous to the reasoning for the Matrix Profile motif discovery algorithm described above, extracted subsequences of $\hat{X}^e$ should start at a point in time that is divisable by two respectively three when $\hat{X}$ was discretized based on the \ac{1d-SAX} respectively \ac{eSAX}, assuming that for $\hat{X}^e$ the points in time start at zero. Also, $l$ should be divisible by two respectively three. \newline
The extracted subsequences of $\hat{X}^e$ are then used as the input for \mbox{Algorithm \ref{alg:brute_force}} \cite{Motif_Definitions}. For each of those, this algorithm collects all extracted subsequences of $\hat{X}^e$ that fulfill the match()-predicate in line 6 of Algorithm \ref{alg:brute_force}. It then returns the largest collection of such subsequences, which is called a motif. Algorithm \ref{alg:brute_force} is invoked iteratively as long as a motif that contains at least two subsequences is found, since all subsequences that are already included in a motif cannot be selected for future motifs. When Algorithm \ref{alg:brute_force} does not find a motif anymore, the subsequences of $X$ that correspond to the found motifs can be retrieved. \newline
The match()-predicate in line 6 of Algorithm \ref{alg:brute_force} iteratively compares the currently selected subsequence with all inputted subsequences \cite{Motif_Definitions}. Let $S_1$ and $S_2$ be two subsequences of $\hat{X}^e$ that shall be compared. Then, the match()-predicate first checks if $D(S_1,S_2) \leq r$ holds, where $D$ is a given Minkowski distance and $r \geq 0$ is a given similarity distance. If this first check is passed, it then compares each pair of corresponding encoded alphabet symbols in $S_1$ and $S_2$ at the same point in time. For each of these pairs, it checks if their absolute difference is not too large based on a given threshold $abs \textunderscore diff \geq 0$. For the last check, the match()-predicate checks if $H(S_1,S_2) \leq h$ holds, where $H$ is the Hamming distance and $h \geq 0$ is a given threshold. \newline
\SetKw{In}{in}
\begin{center}
\begin{algorithm}[H]
  \SetAlgoLined
  \LinesNumbered
  \DontPrintSemicolon
  \KwIn{\text{subsequences} \tcp*[f]{extracted subsequences of $\hat{X}^e$} \newline
  		\text{thresholds for match()-predicate in line 6}}
  \KwOut{\text{largest\_motif}}
  
  \text{largest\_motif} $\leftarrow ()$\;
  \text{largest\_count} $\leftarrow 0$\;
  
  \For{\upshape \text{curr\_subsequence} \In \text{subsequences}}{
  	\text{curr\_motif} $\leftarrow ()$\;
  	\For{\upshape \text{subsequence} \In \text{subsequences}}{
  		\If{\upshape \text{match(curr\_subsequence, subsequence)}}{
			\text{curr\_motif.add(subsequence)}\;
  		}
  	}
  	\If{\upshape \text{curr\_motif.size() > largest\_count}}{
  		\text{largest\_motif} $\leftarrow$ \text{curr\_motif}\;
		\text{largest\_count} $\leftarrow$ \text{curr\_motif.size()}\;
  	}
  }
  \text{return largest\_motif}\;
  
  \caption[Brute Force Motif Discovery - Algorithm]{This brute force algorithm fixes one of the inputted subsequences in each iteration and employs a nested loop over all inputted subsequences \cite{Motif_Definitions}. It finds the largest collection from the inputted subsequences that fulfill the match()-predicate in line 6 with respect to the corresponding fixed subsequence.}
  \label{alg:brute_force}
\end{algorithm}
\end{center}















