\section{Matrix Profile} \label{description_matrix_profile}
Let $X = x_1, ..., x_N$ be a standardized time series of length $N \geq 1$. Then, the first step of the Matrix Profile motif discovery algorithm is to discretize $X$ based on one of the time series discretization algorithms described in Chapter \ref{chap:ts_discretization}. Let $\hat{X}$ be the resulting discretized time series corresponding to $X$. The second step is to encode the alphabet symbols that represent $\hat{X}$ based on the alphabet size $a \geq 2$ used for discretization. For this, the encoding $e(\alpha_j) = j-1 \ (1 \leq j \leq a)$ is applied on the alphabet symbols $\alpha_j$ that represent $\hat{X}$. Note that for the \ac{1d-SAX}, the discretized means and discretized slopes need to be encoded based on the size of the respective alphabet. Let $\hat{X}^e$ be the encoded discretized time series that results from applying the encoding on each alphabet symbol that represents $\hat{X}$. \newline
In the next step, the Matrix Profile procedure is applied on $\hat{X}^e$ \cite{Matrix_Profile}. With this procedure, the pairwise distances between any two subsequences of $\hat{X}^e$ based on a given length $l \geq 1$ of the subsequences and a given Minkowski distance $D$ are computed. Compared to other procedures that compute such pairwise distances (e.g. brute force), the Matrix Profile procedure provides a more efficient computation \cite{Matrix_Profile}. However, it only works on numerical data, which is the reason $\hat{X}$ needs to be encoded. \newline
Based on the computed pairwise distances between any two subsequences of $\hat{X}^e$, the pairwise mutually nearest neighbors are computed. Two subsequences $S_1$ and $S_2$ of $\hat{X}^e$ are pairwise mutually nearest neighbors if both have their smallest distance to the respective other according to the computed pairwise distances. All pairwise mutually nearest neighbors are then sorted in an ascending order with respect to the pairwise distance. \newline
Based on this sorting, the idea of the Random Projection motif discovery algorithm described above is applied \cite{Random_Projection}. Matching subsequences of $X$ are likely to have a relatively small pairwise distance. Therefore, the next step is to use this sorting as a filter that indicates which subsequences of $\hat{X}^e$ shall be examined for finding matching subsequences of $X$ and which not. First, the two subsequences of $\hat{X}^e$ that represent the pairwise mutually nearest neighbors with the smallest pairwise distance are retrieved. Let $S_{1}^*$ and $S_{2}^*$ be those subsequences. If $D(S_{1}^*,S_{2}^*) \leq r$ holds for a given similarity distance $r \geq 0$, then $S_{1}^*$ and $S_{2}^*$ are declared a tentative motif. This tentative motif is expanded to a motif by adding each other subsequence $S_3$ with length $l$ of $\hat{X}^e$ that fulfills $D(S_{1}^*,S_{3}) \leq r$ or $D(S_{2}^*,S_{3}) \leq r$. Such a subsequence $S_3$ is found by computing the pairwise distance between any other subsequence with length $l$ of $\hat{X}^e$ and $S_{1}^*$ respectively $S_{2}^*$ based on the Matrix Profile procedure \cite{Matrix_Profile}. \newline
The final motif is then the one that contains $S_{1}^*$ and $S_{2}^*$ along with all other subsequences $S_3$ with length $l$ of $\hat{X}^e$ that fulfill $D(S_{1}^*,S_{3}) \leq r$ or $D(S_{2}^*,S_{3}) \leq r$ (see Figure \ref{fig:two_circles}) \cite{Random_Projection}. Based on such a final motif, the corresponding subsequences of $X$ can be retrieved. This motif building procedure is done iteratively, while all subsequences of $\hat{X}^e$ that are already included in a motif cannot be selected for future motifs. In each iteration, the two subsequences of $\hat{X}^e$ that represent the pairwise mutually nearest neighbors with the smallest pairwise distance in the current iteration are retrieved until no more subsequences of $\hat{X}^e$ are available to be included in a motif. \newline
Note that for the \ac{1d-SAX} and \ac{eSAX}, a subsequence of $X$ extracted by the \ac{PAA} transforms into two respectively three encoded alphabet symbols in $\hat{X}^e$. Therefore, assuming that for $\hat{X}^e$ the points in time start at zero, subsequences of $\hat{X}^e$ that are used for pairwise distance computations should start at a point in time that is divisible by two respectively three. Moreover, $l$ should also be divisable by two respectively three. These two actions ensure comparability between the subsequences of $\hat{X}^e$ when computing pairwise distances. Otherwise, they would not correspond to the subsequences of $X$ extracted by the \ac{PAA} for the \ac{1d-SAX} and \ac{eSAX}.
\begin{figure}[htb]
\centering
\includegraphics[width=0.8\textwidth]{pattern_recognition/motif_discovery/matrix_profile/two_circles.pdf}
\caption[Matrix Profile - Motif]{A visual intuition for a motif that is discovered by the Matrix Profile motif discovery algorithm \cite{Random_Projection}. In this plot, $D(S_{1}^*,S_{2}^*) = r$ is indicated as an example. Moreover, the red dots indicate subsequences $S_3$ of $\hat{X}^e$ that fulfill $D(S_{1}^*,S_{3}) \leq r$ or $D(S_{2}^*,S_{3}) \leq r$ and are therefore included in the motif.}
\label{fig:two_circles}
\end{figure}







