\newpage
\subsection{1d-Symbolic Aggregate Approximation}
Similar to the \ac{eSAX}, the \ac{1d-SAX} discretization algorithm uses a modified version of the \ac{PAA} in order to compute one additional characteristic value of each subsequence along with the mean \cite{1d-SAX}. This additional value shall also capture information about the trend of the points of a subsequence. \newline
But, in comparison to the \ac{eSAX} and the \ac{SAX}, this additional value is not discretized based on the same assumption that is made for the discretization of the minimum, mean, and maximum of the points of a subsequence \cite{1d-SAX}.
\subsection*{Main Procedure}
Let $X = x_1, ..., x_N$ be a standardized time series of length $N \geq 1$ that follows the standard normal distribution $X \sim \mathcal{N}(0,1)$. The first step of discretizing $X$ based on the \ac{1d-SAX} is to apply a modified version of the PAA on $X$ \cite{1d-SAX}. In addition to computing the mean, the trend of the points of each subsequence extracted by the sliding window is computed. This trend is computed based on the slope of the linear regression of the points of the respective subsequence. Let  $S_i = s_{i}^1, ..., s_{i}^w$ be the $i$-th $(1 \leq i \leq n)$ extracted subsequence of $1 \leq n \leq N$ extracted subsequences from $X$ with a window length of $w \geq 1$. Further, let $T_i = t(s_{i}^1), ..., t(s_{i}^w)$ be the corresponding points in time within $X$. For the \ac{1d-SAX}, only the slope $\tilde{s}_i$ of the linear regression of the points of the $i$-th extracted subsequence is needed. This can be computed based on the closed form of the ordinary least squares \mbox{estimation \cite{1d-SAX}:}
\begin{equation}
\tilde{s}_i = \frac{\sum_{j=1}^{w}(t(s_{i}^j)-\overline{T}_i)(s_{i}^j - \overline{S}_i)}{\sum_{j=1}^{w}(t(s_{i}^j)-\overline{T}_i)^2} \stackrel{\mathrm{\overline{S}_i = 0}}{=} \frac{\sum_{j=1}^{w}(t(s_{i}^j)-\overline{T}_i)s_{i}^j}{\sum_{j=1}^{w}(t(s_{i}^j)-\overline{T}_i)^2},
\label{eq:slope}
\end{equation}
where $\overline{S}_i$ and $\overline{T}_i$ represent the means of $S_i$ and $T_i$, respectively. Note that $\overline{S}_i = 0$, because the linear regression is computed within a sliding window. Therefore, the mean of the points of the respective extracted subsequence by the sliding window is used as the $x$-axis and all these points are normalized with respect to this $x$-axis (see Figure \ref{fig:1d-SAX}). \newline
Thus, the resulting \ac{PAA} representation of this modified version can be represented by $X' = (\overline{x}_1, \tilde{s}_1), ..., (\overline{x}_n, \tilde{s}_n)$, where $(\overline{x}_i, \tilde{s}_i)$ is the mean and the slope of the $i$-th extracted subsequence, respectively. \newline
In the next step of the \ac{1d-SAX}, these computed means along with the slopes are then discretized \cite{1d-SAX}. The discretization of the means and the slopes is done separately. While the means are discretized analogous to the \ac{SAX} discretization based on Equation \ref{eq:SAX_Discretization}, the slopes are discretized based on a different assumption. \newline
It is assumed that the slope values follow a Gaussian distribution with mean 0 variance $\sigma^{2}_w$, where $\sigma^{2}_w$ is a decreasing function of the window length $w$ \cite{1d-SAX}. The discretization of the slope values is then based on the quantiles of this Gaussian distribution $\mathcal{N}(0,\sigma^{2}_w)$. Analogous to the \ac{SAX} discretization, these quantiles are then used as breakpoints for discretizing the slope values based on Equation \ref{eq:SAX_Discretization} \cite{1d-SAX}. Note that due to the separate discretization of means and slopes, two possibly different alphabets for discretization can be used \cite{1d-SAX}. For example, the means can be discretized based on six alphabet symbols, while the discretization of the slopes is based on four alphabet symbols. \newline
The question that now remains is what function $\sigma^{2}_w$ should be used to describe the variance of the Gaussian distribution that is used for discretizing the slopes. Based on the literature, it is recommended to use $\sigma^{2}_w = \frac{0.03}{w}$ \cite{1d-SAX}. Empirical analyses with time series that follow Gaussian distributions found that with this function the Gaussian distribution $\mathcal{N}(0,\sigma^{2}_w)$ best describes the distribution of the computed slopes. Note that for any window length $w \geq 1$, the values of such a Gaussian distribution are more concentrated around the mean $0$ compared to the standard normal distribution, because it is $\sigma^{2}_w < 1$ (see Figure \ref{fig:1d-SAX_Breakpoints}). \newline
Applying the described discretizations for the \ac{PAA} representation $X'$, results in the discretized \ac{1d-SAX} representation of the original time series $X$ that can be represented by $\hat{X} = (\hat{x}_1, \hat{s}_1),$ $..., (\hat{x}_n, \hat{s}_n)$, where $(\hat{x}_i, \hat{s}_i)$ represents the separately discretized mean and slope of the $i$-th extracted subsequence, respectively. \newline
Note that for the evaluation in this thesis (see Chapter \ref{evaluation_chapter}), this representation is used as the \ac{1d-SAX} representation of the original time series. However, this representation deviates from the one proposed in the literature \cite{1d-SAX}. In the literature, the separately discretized means and slopes are represented by bit strings and then interleaved. Consider, for example, $\hat{x}_i := \tilde{0}\tilde{1}$ and $\hat{s}_i := 10$. Then, the interleaving would result in one bit string that represents $\hat{x}_i$ and $\hat{s}_i$ together: $\tilde{0}1\tilde{1}0$. This representation is not used in order to be consistent with the representations of the other evaluated discretization algorithms to have a standardized representation across these algorithms where a discretized value is represented by one alphabet symbol.
\begin{figure}[htb]
\centering
\includegraphics[width=0.8\textwidth]{discretization/one_d_sax/one_d_sax.pdf}
\caption[1d-Symbolic Aggregate Approximation - Mean \& Slope]{For the \ac{1d-SAX}, the slope within a subsequence is computed with a linear regression \cite{1d-SAX}. Both, the mean and slope per extracted subsequence are discretized based on Equation \ref{eq:SAX_Discretization}. The discretized \ac{1d-SAX} representation for the original time series in this plot could be \mbox{\texttt{c}\texttt{b} ... \texttt{d}\texttt{b} ... \texttt{d}\texttt{a} ... \texttt{d}\texttt{c} ... \texttt{a}\texttt{c}} ... . Compared to the discretized \ac{SAX} representation \texttt{c} ... \texttt{d} ... \texttt{d} ... \texttt{d} ... \texttt{a} ..., the \ac{1d-SAX} representation captures information about the trend of the points of a subsequence. Note that for visual clarity only every second extracted subsequence is shown in this plot.}
\label{fig:1d-SAX}
\end{figure}
\subsection*{Time Complexity}
For the modified version of the \ac{PAA}, the slope of the linear regression for an extracted subsequence can be computed along with the mean. Using the closed form of the ordinary least squares estimation described by Equation \ref{eq:slope}, the slope can be computed in $\mathcal{O}(w) = \mathcal{O}(1)$ for a fixed window length $w$. Therefore, the time complexity of the modified version of the \ac{PAA} remains $\mathcal{O}(N)$. \newline
Since for the \ac{1d-SAX} two values need to be discretized for each of the $n$ extracted subsequences, the time complexity for the discretization process is $\mathcal{O}(2n \cdot log_{2}(a-1))$ compared to $\mathcal{O}(n \cdot log_{2}(a-1))$ for the \ac{SAX}, where $a \geq 2$ is the fixed alphabet size used for discretizing. \newline
Therefore, the total time complexity of the \ac{1d-SAX} with the time complexity of the \ac{PAA} is $\mathcal{O}(N) + \mathcal{O}(2n \cdot log_{2}(a-1)) = \mathcal{O}(N) + \mathcal{O}(n) = \mathcal{O}(N)$, because it is $N \geq n$.
\newpage
\begin{figure}[htb]
\centering
\includegraphics[width=0.8\textwidth]{discretization/one_d_sax/breakpoints.pdf}
\caption[1d-Symbolic Aggregate Approximation - Breakpoints]{The means of the extracted subsequences are discretized based on breakpoints that are the quantiles of the standard normal distribution as on the left plot \cite{1d-SAX}. Compared to that, the slopes of the extracted subsequences are discretized based on breakpoints that are the quantiles of a Gaussian distribution $\mathcal{N}(0,\frac{0.03}{w})$ with window size $w$ \cite{1d-SAX}. The right plot shows such a Gaussian distribution for $w = 5$. Further, the discretization granularity controlled by the alphabet size $a \geq 2$ can be different. Based on this figure, the means are discretized with $a = 5$ alphabet symbols, while the slopes are discretized with $a = 3$ alphabet symbols.}
\label{fig:1d-SAX_Breakpoints}
\end{figure}
