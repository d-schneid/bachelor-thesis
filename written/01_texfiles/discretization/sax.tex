\subsection{Symbolic Aggregate Approximation}
The \ac{SAX} discretization algorithm applies amplitude discretization to the \ac{PAA} representation to obtain the discretized \ac{SAX} representation of the original time series \cite{SAX_Lin}. Due to the fixed breakpoints discussed above, every \ac{PAA} representation is discretized based on the same discretization intervals.
\subsection*{Main Procedure}
Let $X = x_1, ..., x_N$ be a standardized time series of length $N \geq 1$. Assume that $X$ follows the standard normal distribution $X \sim \mathcal{N}(0,1)$. Further, define $P := \{\frac{j}{a} \mid j \in \{1, ..., a-1\}\}$ for a given natural number $a \geq 2$. Applying the quantile function of the standard normal distribution to the probabilities in $P$ results in the $a$-quantiles $\beta_j \ (1 \leq j \leq a-1)$ of the standard normal distribution. The ascending sorted list of these $a$-quantiles $B := (\beta_0, \beta_1, ..., \beta_{a-1}, \beta_a)$ with $\beta_0 = -\infty$ and $\beta_a = +\infty$ are called breakpoints and $a$ is called alphabet size \cite{SAX_Lin_first}. Define $A := \{\alpha_j \mid j \in \{1, ..., a\}\}$ as the corresponding alphabet with $a$ symbols (e.g. letters from the Latin alphabet). \newline
Discretizing $X$ now first requires its transformation into its \ac{PAA} representation $\overline{X} = \overline{x}_1, ..., \overline{x}_n$ with $1 \leq n \leq N$. Then, the discretized \ac{SAX} representation $\hat{X} = \hat{x}_1, ..., \hat{x}_n$ of $X$ is obtained by mapping the means $\overline{x}_i$ to alphabet symbols $\alpha_j$ \cite{SAX_Lin_first}:
\begin{equation}
\hat{x}_i = \alpha_j \iff \beta_{j-1} \leq \overline{x}_i < \beta_j \qquad (1 \leq i \leq n, \; 1 \leq j \leq a)
\label{eq:SAX_Discretization}
\end{equation}
Hence, all means $\overline{x}_i$ of the \ac{PAA} representation of $X$ that are smaller than the breakpoint $\beta_1$ are mapped to the alphabet symbol $\alpha_1$, all means that are greater than or equal to the breakpoint $\beta_1$ and smaller than the breakpoint $\beta_2$ are mapped to the alphabet symbol $\alpha_2$ and so on (see Figure \ref{fig:SAX_discretization}) \cite{SAX_Lin_first}. Therefore, the discretization intervals along the amplitude are defined by $[\beta_{j-1},\beta_j) \ (1 \leq j \leq a)$. \newline
Further, note that the breakpoints $\beta_j$ split the standard normal distribution into $a$ areas, each containing $\frac{1}{a}$ of the total area under the curve, given their definition as $a$-quantiles of the standard normal distribution \cite{SAX_Lin_first}.
\begin{figure}[htb]
\centering
\includegraphics[width=0.8\textwidth]{discretization/sax/discretization.pdf}
\caption[Symbolic Aggregate Approximation - Discretization]{The original time series is discretized based on its PAA representation with an alphabet size of $a = 4$. It is assumed that the original time series follows the standard normal distribution $\mathcal{N}(0,1)$. Based on this assumption the discretization intervals along its amplitude are determined \cite{SAX_Lin_first}. Using lowercase letters from the Latin alphabet for the alphabet symbols, the resulting discretized representation of the original time series is: \texttt{bcddcdbaaaabcbbbbaa}.}
\label{fig:SAX_discretization}
\end{figure}
\newpage
\subsection*{Role of the Alphabet Size as Input Parameter} \label{parameter_alphabet_size}
From the definition of the breakpoints as $a$-quantiles of the standard normal distribution, it follows that the alphabet size $a$ is equal to the number of discretization intervals. Furthermore, if the alphabet size $a$ increases the distance $\beta_j - \beta_{j-1}$ between two adjacent breakpoints decreases, because each discretization interval $[\beta_{j-1}, \beta_j)$ covers $\frac{1}{a}$ of the area under the curve of the standard normal distribution \cite{SAX_Lin_first}. Therefore, the range of continuous values that each alphabet symbol covers becomes narrower when the alphabet size $a$ increases. \newline
Assuming that the means $\overline{x}_i \ (1 \leq i \leq n)$ of the \ac{PAA} representation $\overline{X}$ follow the standard normal distribution, each discretization interval contains $\frac{1}{a}$ of the total number of means $\overline{x}_i$. Thus, the number of means in each discretization interval decreases when the alphabet size $a$ increases. \newline
Hence, the alphabet size $a$ controls the granularity of the discretization of the means $\overline{x}_i$ \cite{SAX_Lin}. Meaning that the greater the alphabet size, the more likely it is that each different mean is assigned a different alphabet symbol when discretizing. In other words, it is more likely that each different mean is assigned its own unique alphabet symbol. \newline
This effect can be even more extreme when considering a small window length used in the \ac{PAA}. For example, consider a window length of $w = 1$. Then it is $\overline{X} = X$, meaning that each point in $X$ is its own mean. Further, suppose an alphabet size of $\lim_{a \to +\infty} a$. Assuming that $X$ does not contain any duplicate values, it follows that each point $x_i \ (1 \leq i \leq N)$ of $X$ is assigned its own unique alphabet symbol when discretizing. \newline
All in all, increasing the alphabet size results in a more granular discretization of the means of the \ac{PAA} representation and vice versa (see Figure \ref{fig:SAX_alphabet_size}). However, increasing the alphabet size also results in increased memory requirements for storing the \ac{SAX} representation $\hat{X}$, because the more alphabet symbols used the greater the memory requirement per symbol. Therefore, a trade-off dependent on the alphabet size exists between the granularity of the discretization and the memory requirements of the \ac{SAX} representation of the original time series \cite{SAX_Lin}.
\begin{figure}[htb]
\centering
\includegraphics[width=0.8\textwidth]{discretization/sax/sax_alphabet_density.pdf}
\caption[Symbolic Aggregate Approximation - Effect of Alphabet Size]{In the left plot, the breakpoints for an alphabet size of $a = 5$ are shown. In the right plot, the breakpoints for an alphabet size of $a = 8$ are shown. For instance, the range of continuous values the alphabet symbol \textcolor{red}{c} covers is narrower for $a = 5$ compared to $a = 8$. Thus, the discretization shown in the right plot is more granular by using more alphabet symbols.}
\label{fig:SAX_alphabet_size}
\end{figure}
\subsection*{Distance Measures} \label{dist_measure_mindist}
One usage of distance measures in the context of time series is to measure the similarity between a given time series and each time series stored in a database in order to retrieve similar time series from it \cite{Survey_Esling}. There are two reasons why such a procedure is more time efficient with \ac{SAX} representations than  with the corresponding original time series. \newline
The first reason is that time series databases storing original time series are more likely to need to be stored on disk while those storing \ac{SAX} representations are more likely to be held in main memory due to less memory requirements \cite{SAX_Lin}. Therefore, accessing \ac{SAX} representations is more likely to be faster than accessing original representations. \newline
The second reason is that the time efficiency of distance computations between two time series is dependent on the number of points these time series have. Therefore, distance computations between \ac{SAX} representations are faster than between the corresponding original time series due to less number of points respectively alphabet symbols \cite{SAX_Lin}. \newline
For these two reasons, the following time efficient procedure can be applied based on \ac{SAX} representations \cite{SAX_Lin, Faloutsos_Bounding_Lemma}. \newline
Suppose two time series databases $DB := \{X \mid X \ time \ series \ of \ length \ N \geq 1\}$ and $DB^{SAX} := \{\hat{X} \mid X \in DB\}$ where each time series $X$ is stored in its \ac{SAX} representation $\hat{X}$ of length $ 1 \leq n \leq N$.  Further, assume that $DB$ is stored on disk while $DB^{SAX}$ is held in main memory due to less memory needed for $DB^{SAX}$ than for $DB$. \newline
Now given a time series $X^*$ of length $N$, the set of time series 
\begin{equation*}
DB^* := \{X \mid X \in DB, \ D_2(X^*, X) \leq r\}
\end{equation*}
shall be found, where $D_2$ is the Euclidean distance between two original time series and $r \geq 0$ is a given real number distance. The Euclidean distance is assumed to be the true distance between two original time series and is one possible distance measure for this procedure \cite{Survey_Esling}. \newline
The first step for finding $DB^*$ is to discretize $X^*$ based on the \ac{SAX} in order to be able to compare its \ac{SAX} representation $\hat{X}^*$ with the \ac{SAX} representations in $DB^{SAX}$. \newline
The next step is to define a distance measure $MINDIST$ that measures the distance between any two \ac{SAX} representations and lower-bounds the Euclidean distance for the corresponding original time series (see Figure \ref{fig:Euclidean_Mindist}):
\begin{equation}
MINDIST(\hat{X}^*,\hat{X}) \leq D_2(X^*,X)
\label{eq:lowerBounding}
\end{equation}
With $MINDIST$, the \ac{SAX} representations in $DB^{SAX}$ can be filtered by discarding those that definitely do not fullfill $D_2(X^*,X) \leq r$. Then, the resulting filtered database is $DB^{SAX'} := \{\hat{X} \mid \ X \in DB, \ MINDIST(\hat{X}^*,\hat{X}) \leq r\}$. This filtering property of $MINDIST$ follows from Equation \ref{eq:lowerBounding}, because each $\hat{X} \in DB^{SAX}$ with $MINDIST(\hat{X}^*,\hat{X}) > r$ can be neglected, since it is:
\begin{equation}
MINDIST(\hat{X}^*,\hat{X}) > r \implies D_2(X^*,X) > r
\end{equation}
Equation \ref{eq:lowerBounding} also guarantees no false dismissals during this filtering process. Hence, there is no $\hat{X} \in DB^{SAX}$ that is filtered out, but fulfills $D_2(X^*,X) \leq r$. This property of $MINDIST$ is known as the Lower Bounding Lemma \cite{Faloutsos_Bounding_Lemma}. \newline
However, after this filtering process there can be still some $\hat{X} \in DB^{SAX'}$ where $D_2(X^*,X) > r$ holds, because $MINDIST$ underestimates the Euclidean distance according to Equation \ref{eq:lowerBounding}. \newline
Hence, the last step to find $DB^*$ is to filter these false positives out of $DB^{SAX'}$. This is done by retrieving the original time series $X$  of all $\hat{X} \in DB^{SAX'}$ from $DB$ and computing $D_2(X^*,X)$. $DB^*$ then results from keeping each $X \in DB$ that fulfills $D_2(X^*,X) \leq r$. \newline
Note that only during this last step, the original time series $X$ are retrieved from $DB$ that is stored on disk. Further, not every $X \in DB$ is retrieved, but only those that are in the filtered $DB^{SAX'}$. Before this last step, the distances are computed on the \ac{SAX} representations $\hat{X}$ based on $MINDIST$. These \ac{SAX} representations are held in main memory and are represented by less number of alphabet symbols compared to the number of points of the original time series. \newline
The question that now remains is what distance measure $MINDIST$ that lower-bounds the Euclidean distance shall be used. One important characteristic of the \ac{SAX} is that it provides such a lower bounding distance measure. It is defined as \cite{SAX_Lin}:
\begin{equation}
MINDIST(\hat{X},\hat{Y}) := \sqrt{\frac{N}{n}}\sqrt{\sum_{i=1}^{n} (dist(\hat{x}_i, \hat{y}_i))^2}
\label{eq:mindist}
\end{equation}
where $\hat{X} = \hat{x}_1, ..., \hat{x}_n$ and $\hat{Y} = \hat{y}_1, ..., \hat{y}_n$ are the \ac{SAX} representations with $n$ alphabet symbols of the original time series $X$ and $Y$ of length $N$, respectively. The subfunction $dist$ is defined for $1 \leq i,j \leq a$ as:
\begin{equation}
dist(\alpha_i, \alpha_j) :=
\begin{cases}
0, & \text{if } |i - j| \leq 1 \\
\beta_{\max \{i,j\}-1} - \beta_{\min \{i,j\}}, & \text{otherwise}
\end{cases}
\label{eq:dist}
\end{equation}
where $\alpha_i, \alpha_j$ are alphabet symbols and $\beta_i, \beta_j$ are breakpoints.
\begin{figure}[htb]
\centering
\begin{subfigure}[t]{0.8\textwidth}
\includegraphics[width=\textwidth]{discretization/sax/euclidean_mindist_1.pdf}
\caption{The Euclidean distance is indicated by connecting each pair of points of the original time series $X^*$ and $X$ that belong to the same point in time \cite{SAX_Lin}.}
\label{fig:Euclidean_Mindist_1}
\end{subfigure}
\begin{subfigure}[t]{0.8\textwidth}
\includegraphics[width=\textwidth]{discretization/sax/euclidean_mindist_2.pdf}
\caption{The $MINDIST$ of the corresponding \ac{SAX} representations $\hat{X}^*$ and $\hat{X}$ is indicated by connecting the pairwise symbols whose distance shall be measured \cite{SAX_Lin}.}
\label{fig:Euclidean_Mindist_2}
\end{subfigure}
\vspace*{0.2cm}
\caption[Symbolic Aggregate Approximation - Euclidean vs. MINDIST]{A visual intuition for the $MINDIST$ and the Euclidean distance. The $MINDIST$ shall lower-bound the Euclidean distance: $MINDIST(\hat{X}^*, \hat{X}) \leq D_2(X^*, X)$ \cite{SAX_Lin}.}
\label{fig:Euclidean_Mindist}
\end{figure}
\subsection*{Time Complexity}
The first step of the \ac{SAX} involves the computation of the breakpoints. A static lookup table that contains the $a$-quantiles of the standard normal distribution can be used to retrieve the ascending sorted breakpoints for a given alphabet size $a$ \cite{SAX_Lin}. Thus, the time complexity for computing the breakpoints for a given alphabet size is $\mathcal{O}(1)$. Note that the breakpoints $\beta_0 = -\infty$ and $\beta_a = +\infty$ are not actually needed. Therefore, $a$ alphabet symbols can be represented by $a-1$ breakpoints, since it is $|B| = a + 1$. \newline
The final step is to map each of the $n$ means of the \ac{PAA} representation to one alphabet symbol. Based on the retrieved ascending sorted breakpoints, a binary search can be applied for the mapping of each mean. Thus, the time complexity for this step is $\mathcal{O}(n \cdot log_{2}(a-1)) = \mathcal{O}(n)$ for a given alphabet size $a$. \newline
Therefore, the time complexity of the discretization process with the \ac{SAX} is $\mathcal{O}(n)$.
Taken the time complexity of $\mathcal{O}(N)$ of the \ac{PAA} into account, the total time complexity of the \ac{SAX} is $\mathcal{O}(n) + \mathcal{O}(N) = \mathcal{O}(N)$, because it is $N \geq n$ \cite{SFA}.