\subsection{Extended Symbolic Aggregate Approximation}
Similar to the \ac{SAX}, the \ac{eSAX} discretization algorithm applies amplitude discretization to the \ac{PAA} representation in order to obtain the discretized \ac{eSAX} representation of the original time series \cite{E_SAX}. Moreover, the \ac{eSAX} uses the same fixed breakpoints as the \ac{SAX} for computing the discretization intervals along the amplitude. \newline
However, the \ac{eSAX} does not use the same \ac{PAA} version as the \ac{SAX}, but modifies it in order to extract two additional characteristic values from each subsequence that is extracted by the sliding window \cite{E_SAX}. With these two additional characteristic values, the trend of the points of a subsequence shall be captured.
\subsection*{Main Procedure}
Let $X = x_1, ..., x_N$ be a standardized time series of length $N \geq 1$ that follows the standard normal distribution $X \sim \mathcal{N}(0,1)$. The first step of discretizing $X$ based on the \ac{eSAX} is to apply a modified version of the PAA on $X$ \cite{E_SAX}. In addition to computing the mean, the minimum as well as the maximum point of the points in each sliding window are retrieved. Hence, the resulting \ac{PAA} representation of this modified version can be represented by $X' = \{min_1, \overline{x}_1, max_1\}, ..., \{min_n, \overline{x}_n, max_n\}$, where $1 \leq n \leq N$ and $\{min_i, \overline{x}_i, max_i\}$ $(1 \leq i \leq n)$ is the minimum point, mean, and maximum point of the points of the $i$th extracted subsequence by the sliding window, respectively. \newline
In the next step of the \ac{eSAX}, these computed means along with the minima and maxima are then discretized analogous to the \ac{SAX} discretization based on Equation \ref{eq:SAX_Discretization} \cite{E_SAX}. The resulting time series after this step can then be represented by $X'' = \{\hat{min_1}, \hat{x}_1, \hat{max_1}\}, ..., \{\hat{min_n}, \hat{x}_n, \hat{max_n}\}$, where $\{\hat{min_i}, \hat{x}_i, \hat{max_i}\}$ are the alphabet symbols for the minimum point, mean, and maximum point, respectively. Thus, the \ac{eSAX} accounts for extreme points within a subsequence that are middled out in the \ac{SAX} discretization (see Figure \ref{fig:SAX_E_SAX}) \cite{E_SAX}. \newline
However, $X''$ is not the final discretized \ac{eSAX} representation, because the positions of $\hat{min_i}$ and $\hat{max_i}$ in the \ac{eSAX} representation shall correspond to the positions of the corresponding $min_i$ and $max_i$ within $X$ with respect to the point in time they occur \cite{E_SAX}. \newline
Therefore, the last step of the \ac{eSAX} involves sorting each $\{\hat{min_i}, \hat{x}_i, \hat{max_i}\}$ according to the positions in time of the corresponding $\{min_i, \overline{x}_i, max_i\}$ within $X$ \cite{E_SAX}. Thus, the final discretized \ac{eSAX} representation can be represented by $\hat{X} = sort(\{\hat{min_1}, \hat{x}_1, \hat{max_1}\}), ..., sort(\{\hat{min_n}, \hat{x}_n, \hat{max_n}\})$ (see Figure \ref{fig:E_SAX}), where the function $sort$ is described by \cite{E_SAX}:
\begin{algorithmic}
\STATE 1. Compute the positions $t(min_i)$, $t(\overline{x}_i)$, and $t(max_i)$ in time within $X$. The position of $\overline{x}_i$ is computed by 
\begin{center}
$t(\overline{x}_i) = \frac{t(s_i^1) \;+\; t(s_i^w)}{2}$,
\end{center}
where $t(s_i^1)$ and $t(s_i^w)$ is the starting and ending position in time of the $i$th extracted subsequence within $X$ based on a window length of $w \geq 1$.
\STATE 2. Sort $\hat{min_i}$, $\hat{max_i}$, and $\hat{x_i}$ in ascending order based on the corresponding computed positions in 1.
\STATE 3. Return the sorted values.
\end{algorithmic}
Since $min_i$ and $max_i$ can be arbitrarily located in the $i$th extracted subsequence of $X$, $sort(\{\hat{min_i}, \hat{x}_i, \hat{max_i}\})$ can have $3! = 6$ different return values, namely every permutation of $\{\hat{min_i}, \hat{x}_i, \hat{max_i}\}$ \cite{E_SAX}.
\subsection*{Time Complexity}
In the modified version of the  \ac{PAA} that is used for the \ac{eSAX}, the minimum and maximum for each extracted subsequence can be computed along with the mean in linear time. Therefore, the time complexity of the modified version of the \ac{PAA} remains $\mathcal{O}(N)$. \newline
Since for the \ac{eSAX} three values need to be discretized for each of the $n$ extracted subsequences, the time complexity for the discretization process is $\mathcal{O}(3n \cdot log_{2}(a-1))$ compared to $\mathcal{O}(n \cdot log_{2}(a-1))$ for the \ac{SAX}, where $a \geq 2$ is the fixed alphabet size used for discretizing. \newline
As a last step, the minimum, maximum, and mean needs to be sorted for each of the $n$ extracted subsequences. This can be done in $\mathcal{O}(3 \cdot log_{2}(3) \cdot n)$. \newline
Therefore, the total time complexity of the \ac{eSAX} with the time complexity of the \ac{PAA} is $\mathcal{O}(N) + \mathcal{O}(3n \cdot log_{2}(a-1)) + \mathcal{O}(3 \cdot log_{2}(3) \cdot n) = \mathcal{O}(N) + \mathcal{O}(n) + \mathcal{O}(n) = \mathcal{O}(N)$, because it is $N \geq n$.
\begin{figure}[htb]
\centering
\includegraphics[width=0.8\textwidth]{discretization/e_sax/e_sax_vs_sax.pdf}
\caption[Extended Symbolic Aggregate Approximation - SAX vs. eSAX]{Based on the alphabet symbols $a$, $b$, $c$ and a window length of $w = 5$, the \ac{SAX} representation in the above plot is \textcolor[rgb]{0,0.39,0}{b} \textcolor[rgb]{0,0.39,0}{b} \textcolor[rgb]{0,0.39,0}{b} \textcolor[rgb]{0,0.39,0}{b} \textcolor[rgb]{0,0.39,0}{b} \textcolor[rgb]{0,0.39,0}{b} for the original time series. On the other hand, the \ac{eSAX} representation in the below plot does not middle out the extreme points within a subsequence and captures the actual pattern of the original time series \cite{E_SAX}: \textcolor[rgb]{0,0,0.39}{a}\textcolor[rgb]{0,0.39,0}{b}\textcolor{orange}{c} \textcolor{orange}{c}\textcolor[rgb]{0,0.39,0}{b}\textcolor[rgb]{0,0,0.39}{a} \textcolor[rgb]{0,0,0.39}{a}\textcolor[rgb]{0,0.39,0}{b}\textcolor{orange}{c} \textcolor{orange}{c}\textcolor[rgb]{0,0.39,0}{b}\textcolor[rgb]{0,0,0.39}{a} \textcolor[rgb]{0,0,0.39}{a}\textcolor[rgb]{0,0.39,0}{b}\textcolor{orange}{c} \textcolor{orange}{c}\textcolor[rgb]{0,0.39,0}{b}\textcolor[rgb]{0,0,0.39}{a}.}
\label{fig:SAX_E_SAX}
\end{figure}
\begin{figure}[htb]
\centering
\includegraphics[width=0.8\textwidth]{discretization/e_sax/e_sax.pdf}
\caption[Extended Symbolic Aggregate Approximation - Discretization]{For the \ac{eSAX}, the minimum point, mean, and maximum point per extracted subsequence are discretized analogous to the \ac{SAX} discretization based on Equation \ref{eq:SAX_Discretization} \cite{E_SAX}. Thus, with the alphabet symbols $a$, $b$, $c$, $d$  and a window length of $w = 5$, the resulting discretized \ac{eSAX} representation for this plot  is: \mbox{... \textcolor[rgb]{0,0,0.39}{c}\textcolor{orange}{d}\textcolor[rgb]{0,0.39,0}{c} ... \textcolor[rgb]{0,0,0.39}{a}\textcolor[rgb]{0,0.39,0}{c}\textcolor{orange}{d} ... \textcolor{orange}{d}\textcolor[rgb]{0,0.39,0}{d}\textcolor[rgb]{0,0,0.39}{c} ... \textcolor[rgb]{0,0,0.39}{c}\textcolor[rgb]{0,0.39,0}{d}\textcolor{orange}{d} ... \textcolor{orange}{a}\textcolor[rgb]{0,0.39,0}{a}\textcolor[rgb]{0,0,0.39}{a}}. Note that for visual clarity only every second extracted subsequence is shown in this plot.}
\label{fig:E_SAX}
\end{figure}